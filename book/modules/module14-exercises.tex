\begin{exercises}
	\begin{problist}
		\prob Let $\mathcal{T}:\R^{2}\to\R^{2}$ be defined by $\mathcal{T}\mat{x\\y}
		=\matc{3x-y\\x-\tfrac{1}{4}y}$. Find the volume of $\mathcal{T}(C_{2})$.

		\prob Let $\mathcal{S}:\R^{3}\to\R^{3}$ be defined by
		$\mathcal{S}\mat{x\\y\\z}=\matc{2x+y+z\\x-\tfrac{1}{2}y\\z}$. Find the volume
		of $\mathcal{S}(C_{3})$.

		\prob Let $\mathcal{T}:\R^{2}\to\R^{2}$ be defined by $\mathcal{T}\mat{x\\y}
		=\matc{x+2y\\-x-y}$.
		\begin{enumerate}
			\item Draw $\mathcal{E}$ and $\mathcal{T}(\mathcal{E})$ and then determine
				whether $\mathcal{T}$ is orientation preserving or orientation reversing.

			\item Find $\det(\mathcal{T})$.
		\end{enumerate}

		\prob For each linear transformation defined below, find its determinant.
		\begin{enumerate}
			\item $\mathcal{S}:\R^{2}\to\R^{2}$, where $\mathcal{S}$ shortens
				every vector by a factor of $\tfrac{2}{3}$.

			\item $\mathcal{R}:\R^{2}\to\R^{2}$, where $\mathcal{R}$ is rotation
				counter-clockwise by $90^{\circ}$.

			\item $\mathcal{F}:\R^{2}\to\R^{2}$, where $\mathcal{F}$ is
				reflection across the line $y=-x$.

			\item $\mathcal{G}:\R^{2}\to\R^{2}$, where
				$\mathcal{G}(\vec x)=\mathcal{P}(\vec x)+ \mathcal{Q}(\vec x)$
				and where $\mathcal{P}$ is projection onto the line $y=x$ and
				$\mathcal{Q}$ is projection onto the line $y=-\tfrac{1}{2}x$.

			\item $\mathcal{T}:\R^{3}\to\R^{3}$, where
				$\mathcal{T}\mat{x\\y\\z}=\matc{x-y+z\\z+x-\tfrac{1}{3}y\\z}$.

			\item $\mathcal{J}:\R^{3}\to\R^{3}$, where
				$\mathcal{J}\mat{x\\y\\z}=\matc{0\\0\\x+y+z}$.

			\item $\mathcal{K}\circ \mathcal{H}:\R^{2}\to\R^{2}$, where
				$\mathcal{H}\mat{x\\y}=\matc{x+2y\\-x-y}$,

				and $\mathcal{K}\mat{x\\y}=\matc{-x-2y\\x+y}$.
		\end{enumerate}

		\prob Let $A=\mat{2&3\\1&5}$.
		\begin{enumerate}
			\item Use elementary matrices to find $\det(A)$.

			\item Draw a picture of the parallelogram given by the rows of $A$.
				\label{PROBMOD14-rows}

			\item Draw a picture of the parallelogram given by the columns of $A$.
				\label{PROBMOD14-cols}

			\item How do the areas of the parallelograms drawn in parts \ref{PROBMOD14-rows}
				and \ref{PROBMOD14-cols} relate?
		\end{enumerate}

		\prob Let $A=\mat{1&2&0\\0&2&1\\1&2&3}$.
		\begin{enumerate}
			\item \label{Module14-q8} Use elementary matrices to find $\det(A)$.

			\item Find $\det(A^{-1})$.

			\item Find $\det(A^{T})$, and compare your answer with
				\ref{Module14-q8}. Are they the same? Explain.
		\end{enumerate}

		\prob Let $A$ be an $n \times n$ matrix that can be decomposed into the
		product of elementary matrices.
		\begin{enumerate}
			\item What is $\Rank(A)$? Justify your answer.

			\item What is $\Null(A^{-1})$? Justify your answer.
		\end{enumerate}

		\prob Anna and Ella are studying the relationship between determinant and
		volume. In particular, they are studying $\mathcal{S}:\mathbb{R}^{3}\rightarrow
		\mathbb{R}^{3}$ defined by $\mathcal{S}\mat{x \\ y \\ z}=\mat{4x \\ 2z \\ 0}$,
		and $\mathcal{T}:\mathbb{R}^{3}\rightarrow\mathbb{R}^{2}$ defined by
		$\mathcal{T}\mat{x \\ y \\ z}=\mat{2x \\ 8z}$.

		For each conversation below, (a) evaluate Anna and Ella's arguments as \emph{correct},
		\emph{mostly correct}, or \emph{incorrect}; (b) point out where each argument
		makes correct/incorrect statements; (c) give a correct numerical value
		for the determinant or explain why it doesn't exist.
		\begin{enumerate}
			\item \emph{Anna says:}

				Since the image of $C_{3}$ under $\mathcal{S}$ is the
				parallelepiped generated by
				$\mat{4 \\ 0 \\ 0},\mat{0 \\ 0 \\ 0}, and \mat{0 \\ 2 \\ 0}$, which
				is 2-dimensional parallelogram, the volume of
				$\mathcal{S}(C_{3})$ is just the area of this parallelogram, which
				is 8. Thus, $\det(\mathcal{S})=8$.

				\emph{Ella says:}

				$\det(\mathcal{S})$ is undefined, because $\mathcal{S}$ is not
				invertible.

			\item \emph{Anna says:}

				Since the image of $C_{3}$ under $\mathcal{T}$ is the
				parallelepiped generated by $\mat{2 \\ 0}$, $\mat{0 \\ 0}$, and
				$\mat{0 \\ 8}$, which is a parallelogram in $\mathbb{R}^{2}$,
				the signed volume of $\mathcal{T}(C_{3})$ is just the signed
				area of this parallelogram, which is 16. Thus,
				$\det(\mathcal{T})=16$.

				\emph{Ella says:}

				$\det(\mathcal{T})$ is undefined, because $\det(\mathcal{T})$ is
				only defined when the domain and codomain of $\mathcal{T}$ are
				the same.
		\end{enumerate}
		\begin{solution}
			\begin{enumerate}
				\item \emph{Anna's argument is incorrect.}

					\emph{Reason:} Since $\mathcal{S}$ is a linear
					transformation on $\R^{3}$, its determinant is given by the signed
					change of \emph{3-dimensional volume}. Anna's argument is incorrect
					because she considered the 2-dimensional volume of $\mathcal{S}
					(C_{3})$.

					\emph{Ella's argument is incorrect.}

					\emph{Reason:} The determinant is defined for all linear
					transformations from $\R^{n}$ to $\R^{n}$, no matter whether
					it is invertible or not.

					Finally, $\det(\mathcal{S})=0$, because since $\mathcal{S}(C_{3}
					)$ is a \emph{2-dimensional} object in $\R^{3}$, its \emph{3-dimensional
					volume} is $0$. Therefore, $\VolChange(\mathcal{S})=0$, and we
					conclude that $\det(\mathcal{S})=0$.

				\item \emph{Anna's argument is incorrect.}

					\emph{Reason:} The determinant function is only defined for
					linear transformations with same domain and codomain.

					\emph{Ella's argument is correct.}

					Finally, $\det(\mathcal{T})$ is undefined, because the domain
					and codomain of $\mathcal{T}$ are not the same.
			\end{enumerate}
		\end{solution}
	\end{problist}
\end{exercises}
