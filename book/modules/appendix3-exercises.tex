\begin{exercises}
	\begin{problist}
		\prob For each matrix given below, calculate its determinant by (1) row reduction, and (2) using the $2\times 2$ determinant formula.
		\begin{enumerate}
			\item $A=\mat{1 & 0 \\ 2 & 4}$;

			\item $B=\mat{1 & 5 \\ 1 & 5}$;

			\item $C=\mat{1 & 0 \\ 0 & 1}$;

			\item $D=\mat{1 & 1 \\ 0 & 0}$.
		\end{enumerate}
		\prob For each of the following matrices, calculate its determinant by (1)
		using the diagonal trick, and (2) expanding along a suitable column or row.
		\begin{enumerate}
			\item $E=\mat{1 & 4 & 9 \\ 7 & 6 & 5 \\ 2 & 2 & 3}$;

			\item $F=\mat{1 & 0 & 8 \\ 1 & 4 & 8 \\ 1 & 6 & 5}$;

			\item $G=\mat{2 & 0 & 0 \\ 0 & 2 & 0 \\ 0 & 0 & 3}$;

			\item $H=\mat{0 & 0 & 0 \\ 0 & 0 & 0 \\ 0 & 0 & 0}$.
		\end{enumerate}
        \prob For each ordered set given below, determine, using only determinant, if it constitutes a basis of $\R^2$ (or $\R^3$) and, if yes, determine its orientation.
        \begin{enumerate}
          \item $\mathcal{A}=\Set*{\matc{1 \\ 0 \\ 3},\matc{1 \\ 2 \\ 5},\matc{1 \\ -1 \\ 1}}$;
          \item $\mathcal{B}=\Set*{\matc{1 \\ 4 \\ 9},\matc{1 \\ 2 \\ 3},\matc{1 \\ 1 \\ 1}}$;
          \item $\mathcal{C}=\Set*{\matc{4 \\ 2 \\ 4},\matc{4 \\ 2 \\ 0},\matc{2 \\ 1 \\ 6}}$;
          \item $\mathcal{D}=\Set*{\matc{a^2 \\ ab},\matc{ab \\ b^2}}$, where $a,b\in \R$.
        \end{enumerate}
        \prob Suppose you are studying
		with your pal Kokoro. She understands $3\times 3$ formula on expanding
		along the first column, but she is quite unconfident about the formula on
		expanding along an arbitrary column/row. You are going to help her solve
		her confusion.
		\begin{enumerate}
			\item Explain to her, using complete English sentence, why we can expand
				along the second column. (Hint: you will probably need to use a
				theorem from Module \ref{determinant module}.)

			\item Inspired by your argument, Kokoro then developed an argument
				about the invalidity of expanding along the third columns: \emph{If
				I swap the first and the third column, the third column becomes
				the first column and I can apply the formula for expanding along
				the first column. However, this determinant differs from the
				original determinant by a sign. Thus, we cannot expand along the
				third column.} Of course, her argument is wrong. Explain, using complete
				English sentence, why her argument is wrong.

			\item Now, explain to her, using complete English sentence, why we
				can expand along the third columns. (Hint: swap columns in a smart
				way.)

			\item Finally, explain to her why we can expand along a row. (Hint: refer
				to a theorem in Module \ref{determinant module}.)

            \item Kokoro then came up with a question: \emph{can we expand along a diagonal?}
            Please give her your answer (yes/no). If your answer is ``yes'', explain why; if your
            answer is ``no'', provide an example to show that we cannot expand along a diagonal.
		\end{enumerate}
        \prob Let $M=\mat{a & b \\ c & d}$. The \emph{classical adjoint} of $M$, notated
        $M^{\text{adj}}$, is the matrix given by $M^{\text{adj}}=\mat{d & -b \\ -c & a}$. Prove that if $M$ is invertible, then $\displaystyle M^{-1}=\frac{M^{\text{adj}}}{\abs{M}}$.
		
		\prob For each statement below, determine whether it is true or
           false. Justify your answer.
         \begin{enumerate}
           \item A $2\times 2$ matrix $M$ has determinant $1$ if and only if $M=I_2$.
           \item For any $2\times 2$ matrix $M$, $M^{\text{adj}}M=\abs{M}I_2$.
           \item A $3\times 3$ matrix $M$ has determinant $1$ if and only if $\VolChange(\mathcal{T}_{M})$ is equal to 1, where $\mathcal{T}_M$ is the transformation given by $\mathcal{T}_M(\vec x)=M\vec x$.
           \item For any ordered basis $\mathcal{B}=\left\{\matc{a \\ b},\matc{c \\ d}\right\}\subseteq \mathbb{R}^2$,
           $\mathcal{B}$ is positively oriented if and only if $\VolChange(\mathcal{T})=\det(\mathcal{T})$, where $\mathcal{T}$ is the linear transformation given by $\mathcal{T}(\vec x)=\mat{a & c \\ b & d}\vec x$.
         \end{enumerate}
	\end{problist}
\end{exercises} 