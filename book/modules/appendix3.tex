Determinant is useful: it measures the volume distortion of a linear transformation,
it helps us determine if a matrix is invertible of not, and it provides us with a
simple formula for solving $A\vec x=\vec y$ when $A$ is invertible.

Module \ref{determinant module} discusses an algorithm for evaluating determinants
via row reduction. However, for $2\times 2$ and $3\times 3$ determinants, we
have direct formulas. We will explore these formulas in this appendix.

\Heading{Calculating \(2\times 2\) Determinants} For a $2\times 2$ matrix, once we
know its four entries, we can calculate its determinant directly.
\begin{theorem}[$2\times 2$ determinant formula]\index{Determinant!$2\times 2$ formula}
	Let $M=\mat{a & b \\ c & d}$. Then
	\[
		\det(M)=ad-bc.
	\]
\end{theorem}
\begin{proof}
Let $\mathcal{T}_{M}$ be the matrix multiplication mapping induced by $M$, namely
\[
	\mathcal{T}_{M}:\mathbb{R}^{2}\rightarrow\mathbb{R}^{2},\quad\text{ where }
	\mathcal{T}(\vec x)=M\vec x.
\]

Denote by $C_{2}$ the unit 2-cube. If $\Set*{\matc{a \\ c},\matc{b \\ d}}$ is linearly
dependent, then $\dim \Range(\mathcal{T}_{M})<2$, then $\mathcal{T}_{{M}}(C_{2})$
is either a point or a one-dimensional object. In either case it has signed area
$0$, so $\det(\mathcal{T}_{M})=0$. Thus, $\det(M)=0$. Now, the fact that $\Set*{\matc{a \\ c},\matc{b \\ d}}$
is linearly dependent implies that there exists a scalar $\lambda$ such that $\matc
{a \\ c}=\lambda\matc{b \\ d}$. Then $ad-bc=(\lambda b)d-b(\lambda c)=0$. Thus,
$\det(M)=ad-bc$ in the case when column vectors of $M$ are linearly dependent.

Now let's consider when $\Set*{\matc{a \\ c},\matc{b \\ d}}$ is linearly
independent. Suppose further that $\mathcal{T}_{M}$ is orientation preserving.

We first consider the case as illustrated below, namely tips of
$\mathcal{T}_{M}(\xhat)$ and $\mathcal{T}_{M}(\yhat)$ are in the first and
second quadrant respectively.

XXX Figure (A parallelogram generated by two vectors with tips in Quadrant I and
II respectively)

Let $\mathcal{P}=\mathcal{T}_{M}(C_{2})$. Note that
\[
	\mathcal{T}_{M}(\xhat)= \matc{a \\ c},\quad \mathcal{T}_{M}(\yhat)=\matc{b \\ d}.
\]
Then
\[
	\operatorname{Signed~Area}(\mathcal{P})=\operatorname{Area}(\mathcal{P})=(a-b)(c+d)-2\times \frac{1}{2}ac-2\times \frac{1}{2}(-b)d=ad-bc,
\]
which agrees with the formula. Note that you can always calculate the signed area
in a similar way if the tips of two image vectors are in two neighboring quadrants.

The next case is when tips of $\mathcal{T}_{M}(\xhat)$ and $\mathcal{T}_{M}(\yhat
)$ are both in the first quadrant.

XXX Figure (A parallelogram generated by two vectors with tips both in Quadrant
I)

Let $\mathcal{P}=\mathcal{T}_{M}(C_{2})$. Note that
\[
	\mathcal{T}_{M}(\xhat)= \matc{a \\ c},\quad \mathcal{T}_{M}(\yhat)=\matc{b \\ d}.
\]
Then
\[
	\operatorname{Signed~Area}(\mathcal{P})=\operatorname{Area}(\mathcal{P})=(a+b)(c+d)-2\times \frac{1}{2}ac-2\times \frac{1}{2}bd-2bc=ad-bc,
\]
which agrees with the formula. Note that you can always calculate the signed
area in a similar way if the tips of two image vectors are in two neighboring
quadrants.

The next case is when tips of $\mathcal{T}_{M}(\xhat)$ and
$\mathcal{T}_{M}(\yhat)$ are in the first and third quadrant respectively.

XXX Figure (A parallelogram generated by two vectors with tips in Quadrant I and
III respectively)

Let $\mathcal{P}=\mathcal{T}_{M}(C_{2})$. Note that
\[
	\mathcal{T}_{M}(\xhat)= \matc{a \\ c},\quad \mathcal{T}_{M}(\yhat)=\matc{b \\ d}.
\]
Then
\[
	\operatorname{Signed~Area}(\mathcal{P})=\operatorname{Area}(\mathcal{P})=(a-b)(c-d)-2\times \frac{1}{2}ac-2\times \frac{1}{2}(-b)(-d)-2(a)(-d)=ad-bc,
\]
which agrees with the formula. Note that you can always calculate the signed area
in a similar way if the tips of two image vectors are in the second and fourth quadrants
respectively.

The case when $\mathcal{T}_{M}$ is orientation reversing is left to you.\end{proof}
\begin{example}
	We use the $2\times 2$ determinant formula to find $\det(M)$, where
	$M=\mat{1 & 6 \\ 2 & 7}$. Apply the formula, and we get
	\[
		\det(M)=1\times 7-2\times 6=-5.
	\]
\end{example}
Notice that if we apply row reduction algorithm on $M$, we get
\[
	\mat{1 & -6 \\ 0 & 1}\mat{1 & 0 \\ 0 & -\frac{1}{5}}\mat{1 & 0 \\ -2 & 1}M=I_{2}.
\]
Thus,
\[
	M=\mat{1 & 0 \\ 2 & 1}\mat{1 & 0 \\ 0 & -5}\mat{1 & 6 \\ 0 & 1}.
\]
Using theory developed in Module \ref{determinant module}, this gives us that
$\det(M)=-5$. In other words, the result given by the formula agrees with the
result given by row reduction.

\Heading{Calculating $3\times 3$ Determinants (I): The Diagonal Trick} Now we turn
to $3\times 3$ determinants. We still have a manageable formula, although it is less
appealing than the $2\times 2$ formula:
\begin{theorem}[$3\times 3$ determinant formula]\index{Determinant!$3\times 3$ formula}
	Let $M=\mat{a & b & c \\ d & e & f \\ g & h & i}$. Then
	\[
		\det(M)=aei+bfg+cdh-gec-hfa-idb.
	\]
\end{theorem}
You see, this formula is not easy to remember. To help you remember this formula,
here's a trick called the \textit{diagonal trick}\index{Determinant!$3\times 3$ diagonal trick}. Applying this formula, then, is
no more than applying this trick on the matrix.

Now we outline the procedure of calculating $\det\left(\mat{a & b & c \\ d & e & f \\ g & h & i}\right)
$ via this trick.

\begin{enumerate}[leftmargin=1.4cm]
	\item[\bfseries Step 1. ] Copy the matrix, and then copy the \emph{first two columns} to
		the right of the matrix:
		\[
			\begin{bmatrix}
				[ccc|cc] a & b & c & a & b \\
				d          & e & f & d & e \\
				g          & h & i & g & h
			\end{bmatrix}
		\]

	\item[\bfseries Step 2. ] Now multiply all three numbers highlighted in {\color{LimeGreen} green}.
		Similarly, multiply three numbers colored in {\color{NavyBlue} blue} and {\color{CornflowerBlue} cyan}.
		Finally, sum up the three products you get. Call this number $F$.
		\[
			\begin{bNiceArray}{CCC|CC}[code-before = {\cellcolor{LimeGreen!72}{1-1,2-2,3-3}\cellcolor{NavyBlue!72}{1-2,2-3,3-4}\cellcolor{CornflowerBlue!72}{1-3,2-4,3-5}}]
			 a & b & c & a & b \\
             d & e & f & d & e \\
             g & h & i & g & h
			\end{bNiceArray}
		\]
		\[
			F={\color{LimeGreen} aei}+{\color{NavyBlue} bfg}+{\color{CornflowerBlue} cdh}.
		\]

	\item[\bfseries Step 3. ] Multiply all three numbers highlighted in {\color{LimeGreen} green}.
		Similarly, multiply three numbers colored in {\color{NavyBlue} blue} and {\color{CornflowerBlue} cyan}.
		Finally, sum up the three products you get. Call this number $F$.
		\[
			\begin{bNiceArray}{CCC|CC}[code-before = {\cellcolor{LimeGreen!72}{3-1,2-2,1-3}\cellcolor{NavyBlue!72}{3-2,2-3,1-4}\cellcolor{CornflowerBlue!72}{3-3,2-4,1-5}}]
			 a & b & c & a & b \\
             d & e & f & d & e \\
             g & h & i & g & h
			\end{bNiceArray}
		\]
		\[
			G={\color{LimeGreen} gec}+{\color{NavyBlue} hfa}+{\color{CornflowerBlue} idb}
		\]

	\item[\bfseries Step 4. ] Finally, subtract $G$ from $F$:
		\[
			\det(M)=F-G=aei+bfg+cdh-gec-hfa-idb.
		\]
\end{enumerate}
We illustrate this trick with an example.

\begin{example}
	We calculate $\det\left(\mat{1 & 4 & 0 \\ -2 & 3 & 1 \\ 0 & 2 & 1}\right)$ by the diagonal
	trick.

	\[
	\begin{bNiceArray}{CCC|CC}[code-before = {\cellcolor{LimeGreen!72}{1-1,2-2,3-3}\cellcolor{NavyBlue!72}{1-2,2-3,3-4}\cellcolor{CornflowerBlue!72}{1-3,2-4,3-5}}]
	1 & 4 & 0 & 1 & 4 \\
	-2 & 3 & 1 & -2 & 3 \\
	0 & 2 & 1 & 0 & 2
	\end{bNiceArray}
	\]
	\[
		F={\color{LimeGreen} 1 \times 3 \times 1}+{\color{NavyBlue} 4\times 1 \times 0}+{\color{CornflowerBlue}0\times{-2}\times 2}=3.
	\]

	\[
	\begin{bNiceArray}{CCC|CC}[code-before = {\cellcolor{LimeGreen!72}{3-1,2-2,1-3}\cellcolor{NavyBlue!72}{3-2,2-3,1-4}\cellcolor{CornflowerBlue!72}{3-3,2-4,1-5}}]
	1 & 4 & 0 & 1 & 4 \\
	-2 & 3 & 1 & -2 & 3 \\
	0 & 2 & 1 & 0 & 2
	\end{bNiceArray}\]
	\[
		G={\color{LimeGreen}0 \times 3 \times 0}+{\color{NavyBlue}2\times 1 \times 1}+{\color{CornflowerBlue} 1 \times (-2)\times 4}=-6.
	\]
	Thus, $\det\left(\mat{1 & 4 & 0 \\ -2 & 3 & 1 \\ 0 & 2 & 1}\right)=3-(-6)=9$.
\end{example}

\begin{emphbox}[Warning]
The diagonal trick is invalid in $n\times n$ ($n\ge 4$) cases.
\end{emphbox}

\Heading{Calculating $3\times 3$ Determinants (II): Expanding along the First Column}
The second method of calculating $3\times 3$ determinants actually uses $2\times
2$ determinants. This method is called \emph{expanding along the first column}.

\begin{theorem}\index{Determinant!$3\times 3$ cofactor expansion}
	Let $A$ be a $3\times 3$ matrix. Suppose
	$A=\mat{a_{11} & a_{12} & a_{13} \\ a_{21} & a_{22} & a_{23} \\ a_{31} & a_{32} & a_{33}}$.
	Then
	\[
		\det(A)=a_{11}\det\left(\mat{a_{22} & a_{23} \\ a_{32} & a_{33}}\right)-a_{21}\det\left(\mat{a_{12} & a_{13} \\ a_{32} & a_{33}}\right)+a_{31}\det\left(\mat{a_{12} & a_{13} \\ a_{22} & a_{23}}\right).
	\]
\end{theorem}
\begin{proof}
	Using the $2\times 2$ determinant formula, we get
	\begin{align*}
		  & a_{11}\det\left(\mat{a_{22} & a_{23} \\ a_{32} & a_{33}}\right)-a_{21}\det\left(\mat{a_{12} & a_{13} \\ a_{32} & a_{33}}\right)+a_{31}\det\left(\mat{a_{12} & a_{13} \\ a_{22} & a_{23}}\right) \\
		= & a_{11}(a_{22}a_{33}-a_{23}a_{32})-a_{21}(a_{12}a_{33}-a_{13}a_{32})+a_{31}(a_{12}a_{23}-a_{13}a_{22})                                                    \\
		= & a_{11}a_{22}a_{33}+a_{12}a_{23}a_{31}+a_{13}a_{21}a_{32}-a_{31}a_{22}a_{13}-a_{32}a_{23}a_{11}-a_{33}a_{21}a_{12}.
	\end{align*}
	This agrees with the $3\times 3$ determinant formula.
\end{proof}

Before moving on, we give an explanation about the name ``expanding along the first
column''.

The formula involves three determinants. If we delete the first column and first row of $A$. we are left with a $2\times 2$ matrix (below, a highlighted column/row means that this column/row is deleted).
\[
\begin{bNiceMatrix}[ code-before = {\cellcolor{Peach!28}{1-2,1-3,2-1,3-1} \cellcolor{Peach!42}{1-1}}]
a_{11} & a_{12} & a_{13} \\ a_{21} & a_{22} & a_{23} \\ a_{31} & a_{32} & a_{33}
\end{bNiceMatrix}
\]
Note that the determinant of this matrix is exactly the one appeared in the first term of the formula. We can give a similar interpretation to the other determinants appeared in the formula. Thus, our formula can be interpreted as:
\[\det\left(\begin{bNiceMatrix}
a_{11} & a_{12} & a_{13} \\ a_{21} & a_{22} & a_{23} \\ a_{31} & a_{32} & a_{33}
\end{bNiceMatrix}\right)=a_{11}\det\begin{bNiceMatrix}[ code-before = {\cellcolor{Peach!28}{1-2,1-3,2-1,3-1} \cellcolor{Peach!42}{1-1}}]
a_{11} & a_{12} & a_{13} \\ a_{21} & a_{22} & a_{23} \\ a_{31} & a_{32} & a_{33}
\end{bNiceMatrix}-a_{21}\det\begin{bNiceMatrix}[ code-before = {\cellcolor{Peach!28}{2-2,2-3,1-1,3-1} \cellcolor{Peach!42}{2-1}}]
a_{11} & a_{12} & a_{13} \\ a_{21} & a_{22} & a_{23} \\ a_{31} & a_{32} & a_{33}
\end{bNiceMatrix}+a_{31}\det\begin{bNiceMatrix}[ code-before = {\cellcolor{Peach!28}{3-2,3-3,1-1,2-1} \cellcolor{Peach!42}{3-1}}]
a_{11} & a_{12} & a_{13} \\ a_{21} & a_{22} & a_{23} \\ a_{31} & a_{32} & a_{33}
\end{bNiceMatrix}.\]
This is the origin of the name ``expanding along the first column''.

If we let $M_{rs}$ be the determinant of the matrix obtained by deleting the $r$'th
row and $s$'th column, then
\[
	\det(A)=a_{11}M_{11}-a_{21}M_{21}+a_{31}M_{31}.
\]
Using the summation notation, this can be written in a more compact form:
\[
	\det(A)=\sum_{r=1}^{3}(-1)^{r+1}a_{r1}M_{r1}.\footnote{In fact, this can be generalized to determinants of arbitriary size: for any $n
	\times n$ matrix $A$, $\det(A)=\sum_{r=1}^{n}(-1)^{r+1}a_{r1}M_{r1}$. Some textbooks use this as the definition of determinant.}
\]

\begin{example}
	We calculate $\det\left(\mat{1 & 2 & 3 \\ 2 & 3 & 5 \\ 3 & 1 & 4}\right)$ by expanding along
	the first column.

	We first calculate the three determinants:
	\[
		M_{11}=\det\left(\mat{3 & 5 \\ 1 & 4}\right)=3\times 4 - 5\times 1=7;
	\]
	\[
		M_{21}=\det\left(\mat{2 & 3 \\ 1 & 4}\right)=2\times 4 - 3\times 1=5;
	\]
	\[
		M_{31}=\det\left(\mat{2 & 3 \\ 3 & 5}\right)=2\times 5 - 3\times 3=1.
	\]
	Thus,
	\[
		\det\left(\mat{1 & 2 & 3 \\ 2 & 3 & 5 \\ 3 & 1 & 4}\right)=a_{11}M_{11}-a_{21}M_{21}+a_{31}M_{31}=1\times 7=2\times 5+3\times 1=0.
	\]
	Since $\det\left(\mat{1 & 2 & 3 \\ 2 & 3 & 5 \\ 3 & 1 & 4}\right)=0$, we conclude that
	$\Set*{\mat{1 \\ 2 \\ 3},\mat{2 \\ 3 \\ 1},\mat{3\\ 5 \\4}}$ is linearly
	dependent.
\end{example}

\Heading{Calculating $3\times 3$ determinants (III): Expanding along an Arbitrary Column/Row}
You might be wondering what is special for the first column. Can we also expand
along another column?

The answer is: Yes, we can expand along another column. Indeed, we have the
following theorem:
\begin{theorem}
	Let $s$ be a fixed number selected from $\{1,2,3\}$. Then
	\[
		\det(A)=\sum_{r=1}^{3}(-1)^{r+s}a_{rs}M_{rs}.
	\]
\end{theorem}

Finally, we can also expand along a row.
\begin{theorem}
	Let $r$ be a fixed number selected from $\{1,2,3\}$. Then
	\[
		\det(A)=\sum_{s=1}^{3}(-1)^{r+s}a_{rs}M_{rs}.
	\]
\end{theorem}

\begin{example}
	We calculate $\det\left(\mat{1 & 0 & 5 \\ 2 & 0 & 4 \\ 4 & 2 & 0}\right)$ by expanding along
	a suitable column or row.

	Notice that the second column involves more zeros than other columns or rows.
	Thus, it will be better if we expand along the second row.

	\begin{align*}
  \det\left(\mat{1 & 0 & 5 \\ 2 & 0 & 4 \\ 4 & 2 & 0}\right) & =-a_{12}\det\left(\begin{bNiceMatrix}[code-before = {\cellcolor{Peach!28}{1-1,1-3,2-2,3-2} \cellcolor{Peach!42}{1-2}}]
  1 & 0 & 5 \\ 2 & 0 & 4 \\ 4 & 2 & 0
  \end{bNiceMatrix}\right)+a_{22}\det\left(\begin{bNiceMatrix}[code-before = {\cellcolor{Peach!28}{2-1,2-3,1-2,3-2} \cellcolor{Peach!42}{2-2}}]
  1 & 0 & 5 \\ 2 & 0 & 4 \\ 4 & 2 & 0
  \end{bNiceMatrix}\right)-a_{32}\det\left(\begin{bNiceMatrix}[code-before = {\cellcolor{Peach!28}{3-1,3-3,1-2,2-2} \cellcolor{Peach!42}{3-2}}]
  1 & 0 & 5 \\ 2 & 0 & 4 \\ 4 & 2 & 0
  \end{bNiceMatrix}\right) \\
   &=-2\det\left(\mat{1 & 5 \\ 2 & 4}\right)  \\
   &=12.
\end{align*}
\end{example} 