\begin{exercises}
	\begin{problist}
		\prob For each matrix given below, calculate its determinant using both
		row reduction/elementary matrices and the $2\times 2$ determinant
		formula.
		\begin{enumerate}
			\item $\mat{1 & 0 \\ 2 & 4}$

			\item $\mat{1 & 5 \\ 1 & 5}$

			\item $\mat{1 & 0 \\ 0 & 1}$

			\item $\mat{1 & 1 \\ 0 & 0}$
		\end{enumerate}
		\begin{solution}
			\begin{enumerate}
				\item $4$

				\item $0$

				\item $1$

				\item $0$
			\end{enumerate}
		\end{solution}
		\prob For each matrix given below, calculate its determinant using both
		row reduction/elementary matrices and the $3\times 3$ determinant
		formula.
		\begin{enumerate}
			\item $\mat{1 & 0 & 0 \\ 1 & 0 & 2 \\ 1 & 6 & 5}$

			\item $\mat{1 & -4 & 1 \\ 2 & 6 & 5 \\ 2 & 2 & 3}$

			\item $\mat{-1 & 0 & -8 \\ 1 & -3 & -8 \\ 1 & -2 & -1}$

			\item $\mat{2 & 0 & 0 \\ 0 & 2 & 0 \\ 0 & 0 & 3}$

			\item $\mat{0 & 0 & 0 \\ 0 & 0 & 0 \\ 0 & 0 & 0}$
		\end{enumerate}
		\begin{solution}
			\begin{enumerate}
				\item $-12$

				\item $-16$

				\item $21$

				\item $12$

				\item $0$
			\end{enumerate}
		\end{solution}

		\prob For each ordered set given below, use a determinant to decide whether
		it is a right-handed basis, a left-handed basis, or not a basis.
		\begin{enumerate}
			\item $\Set*{\mat{1 \\ -3},\mat{1 \\ 2}}$

			\item $\Set*{\mat{1 \\ -3},\mat{-2 \\ 6}}$

			\item $\Set*{\mat{1 \\ 0 \\ 3},\mat{1 \\ 2 \\ 5},\mat{1 \\ -1 \\ 1}}$

			\item $\Set*{\mat{1 \\ 4 \\ 9},\mat{1 \\ 2 \\ 3},\mat{1 \\ 1 \\ 1}}$

			\item $\Set*{\mat{4 \\ 2 \\ 4},\mat{4 \\ 2 \\ 0},\mat{2 \\ 1 \\ 6}}$
		\end{enumerate}
		\begin{solution}
			\begin{enumerate}
				\item Since $\det\left(\mat{1 & 1 \\ -3 & 2}\right)=5>0$, the ordered set
					$\Set*{\mat{1 \\ -3},\mat{1 \\ 2}}$ is a right-handed basis.

				\item Since $\det\left(\mat{1 & -2 \\ -3 & 6}\right)=0$, the set
					$\Set*{\mat{1 \\ -3},\mat{-2 \\ 6}}$ is not linearly
					independent and so is not a basis.

				\item Since $\det\left(\mat{1 & 1 & 1 \\ 0 & 2 & -1 \\ 3 & 5 & 1}\right)=-2<
					0$, the ordered set
					$\Set*{\mat{1 \\ 0 \\ 3},\mat{1 \\ 2 \\ 5},\mat{1 \\ -1 \\ 1}}$
					is a left-handed basis.

				\item Since $\det\left(\mat{1 & 1 & 1 \\ 4 & 2 & 1 \\ 9 & 3 & 1}\right)=-2<0$,
					the ordered set
					$\Set*{\mat{1 \\ 4 \\ 9},\mat{1 \\ 2 \\ 3},\mat{1 \\ 1 \\ 1}}$
					is a left-handed basis.

				\item Since $\det\left(\mat{4 & 4 & 2 \\ 2 & 2 & 1 \\ 4 & 0 & 6}\right)=0$,
					the set
					$\Set*{\mat{4 \\ 2 \\ 4},\mat{4 \\ 2 \\ 0},\mat{2 \\ 1 \\ 6}}$
					is not linearly independent and so is not a basis.
			\end{enumerate}
		\end{solution}

		\prob Find all values of $a,b\in \R$ so that the ordered set $\Set*{\mat{a^2 \\ ab},\mat{ab \\ b}}$
		is (a) a right-handed basis, (b) a left-handed basis, (c) not a basis.
		\begin{solution}
			Before answering, note that
			$\det\left(\mat{a^2 & ab \\ ab & b}\right)=a^{2}b-a^{2}b^{2}=a^{2}( b
			-b^{2})$.
			\begin{enumerate}
				\item If $\Set*{\mat{a^2 \\ ab},\mat{ab \\ b}}$ is a right-handed
					basis, then
					$\det\left(\mat{a^2 & ab \\ ab & b}\right)=a^{2}( b-b^{2})>0$.
					This implies that $a^{2}$ and $b-b^{2}$ are both nonzero
					and have the same sign. Since $a^{2}\ge 0$, 
					we must have $a^{2}>0$ and $b-b^{2}>0$. The roots of $b-b^{2}$ are
					$b=0$ and $b=1$, so $b-b^{2}>0$ implies $0<b<1$. Therefore,
					$\Set*{\mat{a^2 \\ ab},\mat{ab \\ b}}$ is a right-handed
					basis when $a\ne 0$ and $0<b<1$.

				\item If $\Set*{\mat{a^2 \\ ab},\mat{ab \\ b}}$ is a left-handed
					basis, then
					$\det\left(\mat{a^2 & ab \\ ab & b}\right)=a^{2}( b-b^{2})<0$.
					This implies that $a^{2}$ and $b-b^{2}$ are both nonzero
					and have different signs. Since $a^{2}\ge 0$, this
					implies that $a^{2}>0$ and $b-b^{2}<0$. Roots of $b-b^{2}$ are
					$b=0$ and $b=1$, so $b-b^{2}<0$ implies $b<0$ or $b>1$. Therefore,
					$\Set*{\mat{a^2 \\ ab},\mat{ab \\ b}}$ is a left-handed
					basis when $a\ne 0$ and either $b<0$ or $b>1$.

				\item If $\Set*{\mat{a^2 \\ ab},\mat{ab \\ b}}$ is not a basis,
					then
					$\det\left(\mat{a^2 & ab \\ ab & b}\right)=a^{2}( b-b^{2})=0$.
					This implies that $a^{2}=0$ or $b-b^{2}=0$. Therefore, $\Set*
					{\mat{a^2 \\ ab},\mat{ab \\ b}}$ is not a basis if one of the
					following conditions holds: $a=0$, $b=1$, or $b=0$.
			\end{enumerate}
		\end{solution}

		\prob Let $M=\mat{a & b \\ c & d}$. The \emph{adjugate matrix} (sometimes
		called the \emph{classical adjoint}) of $M$, notated $M^{\text{adj}}$,
		is the matrix given by $M^{\text{adj}}=\mat{d & -b \\ -c & a}$. Prove that
		if $M$ is invertible, then
		$\displaystyle M^{-1}=\frac{M^{\text{adj}}}{\det(M)}$.
		\begin{solution}
			Since
			\[
				\frac{M^{\text{adj}}}{\det{M}}M=\frac{1}{ad-bc}\mat{da-bc & db-bd \\ ac-ca& ad-cb}
			\]\[
				=\mat{1 & 0 \\ 0 & 1}=I_{2\times 2}
			\]
			and
			\[
				M\frac{M^{\text{adj}}}{\det{M}}=\frac{1}{ad-bc}\mat{ad-bc & ba-ab \\ cd-dc& da-cb}
			\]\[
				=\mat{1 & 0 \\ 0 & 1}=I_{2\times 2},
			\]
			we conclude that
			\[
				M^{-1}=\frac{M^{\text{adj}}}{\det(M)}.
			\]
		\end{solution}

		\prob For each statement below, determine whether it is true or false.
		Justify your answer.
		\begin{enumerate}
			\item A $2\times 2$ matrix $M$ has determinant $1$ if and only if $M=
				I_{2\times 2}$.

			\item A $3\times 3$ matrix $M$ has determinant $1$ if and only if $\VolChange
				(\mathcal{T}_{M})$ is equal to 1, where $\mathcal{T}_{M}$ is the
				transformation given by $\mathcal{T}_{M}(\vec x)=M\vec x$.

			\item For vectors $\vec a,\vec b\in \R^{2}$, it is always the case
				that $\det([\vec a|\vec b])=-\det([\vec b|\vec a])$.

			\item For a $2\times 2$ or $3\times 3$ matrix $M$, multiplying a
				single entry of $M$ by $4$ will change $\det(M)$ by a factor of
				$4$.

			\item For a square matrix $A$, it is always the case that
				$\det(A^{T}A)\geq 0$.
		\end{enumerate}
		\begin{solution}
			\begin{enumerate}
				\item False. A counterexample is
					$M=\mat{2 & 0 \\ 0 & \frac{1}{2}}$.

				\item False. A counterexample is
					$M=\mat{-1 & 0 & 0 \\ 0 & -1 & 0 \\ 0 & 0 & -1}$. $M$ does not
					change volume, but it does reverse orientation.

				\item True. Note that $[\vec a|\vec b]$ is just
					$[\vec b|\vec a]$ with its columns swapped. The oriented volume
					of the parallelogram generated by $\vec a$ and $\vec b$ is equal
					to the negative of the oriented volume of the parallelogram generated
					by $(\vec b,\vec a)$. Using Volume Theorem I, we have $\det([\vec
					a|\vec b])=-\det([\vec b|\vec a])$.

				\item False. A counterexample is
					$M=\mat{1 & 0 & 0 \\ 0 & 1 & 0 \\ 0 & 0 & 1}$. We multiply the
					$(1,2)$-entry by $4$ to get another matrix $M'$. Note that
					$M'$ is still $\mat{1 & 0 & 0 \\ 0 & 1 & 0 \\ 0 & 0 & 1 }$, and
					$\det(M)=1=\det(M')\ne 4\det(M)$.

				\item True. Note that
					$\det(A^{T}A)=\det(A^{T})\det(A)=\det(A)\det(A)=\det(A)^{2}\ge
					0$.
			\end{enumerate}
		\end{solution}
	\end{problist}
\end{exercises} 
